%%%%%%%%%%%%%%%%%%%%%%%%%%%%%%%%%%%%%%%%%
% Developer CV
% LaTeX Template
% Version 1.0 (28/1/19)
%
% This template originates from:
% http://www.LaTeXTemplates.com
%
% Authors:
% Jan Vorisek (jan@vorisek.me)
% Based on a template by Jan Küster (info@jankuester.com)
% Modified for LaTeX Templates by Vel (vel@LaTeXTemplates.com)
%
% License:
% The MIT License (see included LICENSE file)
%
%%%%%%%%%%%%%%%%%%%%%%%%%%%%%%%%%%%%%%%%%

%----------------------------------------------------------------------------------------
%	PACKAGES AND OTHER DOCUMENT CONFIGURATIONS
%----------------------------------------------------------------------------------------

\documentclass[9pt]{developercv} % Default font size, values from 8-12pt are recommended

%----------------------------------------------------------------------------------------

\begin{document}

%----------------------------------------------------------------------------------------
%	TITLE AND CONTACT INFORMATION
%----------------------------------------------------------------------------------------

\begin{minipage}[t]{0.45\textwidth} % 45% of the page width for name
	\vspace{-\baselineskip} % Required for vertically aligning minipages
	
	% If your name is very short, use just one of the lines below
	% If your name is very long, reduce the font size or make the minipage wider and reduce the others proportionately
	\colorbox{purple}{{\HUGE\textcolor{white}{\textbf{\MakeUppercase{Андрей}}}}} % First name
	
	\colorbox{black}{{\HUGE\textcolor{white}{\textbf{\MakeUppercase{Солодов}}}}} % Last name
	
	\vspace{6pt}
	
	{\huge Front-end веб-разработчик} % Career or current job title
\end{minipage}
\begin{minipage}[t]{0.275\textwidth} % 27.5% of the page width for the first row of icons
	\vspace{-\baselineskip} % Required for vertically aligning minipages
	
	% The first parameter is the FontAwesome icon name, the second is the box size and the third is the text
	% Other icons can be found by referring to fontawesome.pdf (supplied with the template) and using the word after \fa in the command for the icon you want
	\icon{Phone}{12}{+7 (961) 293 3630}\\
	\icon{At}{12}{\href{mailto:solodov.dev@gmail.com}{solodov.dev@gmail.com}}\\	
\end{minipage}
\begin{minipage}[t]{0.275\textwidth} % 27.5% of the page width for the second row of icons
	\vspace{-\baselineskip} % Required for vertically aligning minipages
	
	% The first parameter is the FontAwesome icon name, the second is the box size and the third is the text
	% Other icons can be found by referring to fontawesome.pdf (supplied with the template) and using the word after \fa in the command for the icon you want
	\icon{Globe}{12}{\href{https://solodov.dev}{solodov.dev}}\\
	\icon{Github}{12}{\href{https://github.com/solodov-dev}{github.com/solodov-dev}}\\
\end{minipage}

\vspace{0.5cm}
%----------------------------------------------------------------------------------------
%	INTRODUCTION, SKILLS AND TECHNOLOGIES
%----------------------------------------------------------------------------------------
\cvsect{О себе}

\begin{minipage}[t]{0.4\textwidth} % 40% of the page width for the introduction text
	\vspace{-\baselineskip} % Required for vertically aligning minipages
Front-end веб-разработчик, решающий поставленные перед ним задачи. Текущие интересы: JavaScript разработка, Linux и графический дизайн. Планы на будущее: изучение Node.js (Express.js) и React.	
  % About
\end{minipage}
\hfill % Whitespace between
\begin{minipage}[t]{0.5\textwidth} % 50% of the page for the skills bar chart
	\vspace{-\baselineskip} % Required for vertically aligning minipages
	\begin{barchart}{5.5}
    \baritem{HTML}{80}
    \baritem{CSS/SASS}{70}
		\baritem{JavaScript}{60}
    \baritem{Vue.js}{60}
		\baritem{Git}{50}
		% \baritem{Webpack}{40}
		% \baritem{Bootstrap}{30}
	\end{barchart}
\end{minipage}

%----------------------------------------------------------------------------------------
%	EXPERIENCE
%----------------------------------------------------------------------------------------
\cvsect{Опыт работы}

\begin{entrylist}
  \entry
      {2019 --  }
      {Front-end разработчик}
      {Фрилансер}
      {Создание front-end на сайтах для частных клиентов.}
\end{entrylist}

\begin{entrylist}
  \entry
      {2015 --  }
      {Преподаватель английского языка и иврита}
      {Языковая школа Goodwin}
      {Проведение групповых и индивидуальных занятий. Подготовка учебных материалов.}
\end{entrylist}

%----------------------------------------------------------------------------------------
% PROJECTS
%----------------------------------------------------------------------------------------
\cvsect{Проекты}

\begin{entrylist}
   \entry
    {01/2020}
    {"ТЦ"\ Парус}
    {\href{https://parus-nalchik.ru}{parus-nalchik.ru}}
    {Сайт для строительного гипермаркета. Сайт написан на PHP с использованием JavaScript для раздела галереи.
\\ \texttt{PHP}\slashsep\texttt{JavaScript}}

 \entry
    {11/2019}
    {Rad.io}
    {\href{https://solodov-dev.github.io/rad-io}{solodov-dev.github.io/rad-io}}
    {Приложение для прослушивания ваших любимых онлайн-трансляций. Возможность сохранения своих любимых радиостанций в
вашем персональном плей-листе. Для данного проекта использовались Vue.js в качестве фреймворка для front-end и Firebase в качестве back-end.
\\ \texttt{Vue.js}\slashsep\texttt{Firebase}\slashsep\texttt{Open Radio API}}

 \entry
    {09/2019}
    {Интерактивная карта России}
    {\href{https://solodov-dev.github.io/russia}{solodov-dev.github.io/russia}}
    {Интерактивная карта создана с использованием технологий SVG и Vue.js. Вы можете узнать об известных людях, животных и местах России.\\ \texttt{SVG}\slashsep\texttt{Vue.js}}
\end{entrylist}

%----------------------------------------------------------------------------------------
%	EDUCATION
%----------------------------------------------------------------------------------------
\cvsect{Образование}

\begin{entrylist}
  \entry
      {04/2019 -- 10/2019}
      {\href{https://www.theodinproject.com}{Онлайн курс по web-разработке}}
      {\href{https://www.theodinproject.com}{The Odin Project}}
      {Курс включает в себя углубленное изучение HTML, CSS, JavaScript, Webpack, npm, JS фреймворк (Vue.js), тестирование с Jest.}

  \entry
      {03/2019 -- 04/2019}
      {\href{https://www.edx.org/course/cs50s-introduction-to-computer-science}{Онлайн курс CS50}}
      {\href{https://www.edx.org/course/cs50s-introduction-to-computer-science}{Гарвардский Университет}}
      {Курс дает базовые знания основ программирования - синтаксис, структуры данных, введение в алгоритмы. Все проекты выполнены на языках С и Python.}

  \entry
      {2010 -- 2013}
      {Бакалавриат}
      {Иерусалимский Университет}
      {Бакалавр гуманитарных наук в области лингвистики и философии.}
\end{entrylist}

%----------------------------------------------------------------------------------------
%	ADDITIONAL INFORMATION
%----------------------------------------------------------------------------------------

\begin{minipage}[t]{0.3\textwidth}
	\vspace{-\baselineskip} % Required for vertically aligning minipages

	\cvsect{Языки}
	
	\textbf{Английский} - свободно\\
	\textbf{Иврит} - свободно\\
	\textbf{Русский} - родной\\
\end{minipage}
\hfill
\begin{minipage}[t]{0.3\textwidth}
	\vspace{-\baselineskip} % Required for vertically aligning minipages
	
	\cvsect{Дополнительные навыки}
  
  \textbf{Photoshop} - высокий \\
  \textbf{Figma} - средний \\ 
  \textbf{\LaTeX} - начинающий \\ 
	
	\end{minipage}
\hfill
\begin{minipage}[t]{0.3\textwidth}
	\vspace{-\baselineskip} % Required for vertically aligning minipages
	
	\cvsect{Интересы}

Люблю читать, изучать новые технологии и кататься на роликах.	
	\end{minipage}
\hfill

%----------------------------------------------------------------------------------------

\end{document}
